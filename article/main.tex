\documentclass{article}

\usepackage{tabularray}
\usepackage{float}
\usepackage{multirow}
\usepackage{array}
\usepackage{graphicx}
\usepackage{codehigh}
\usepackage[normalem]{ulem}
\usepackage{adjustbox}
\usepackage{tabu}
\usepackage{longtable}
\usepackage[para]{threeparttablex}
\usepackage{tabularx}
\usepackage[normalem]{ulem}
\usepackage{makecell}
\usepackage{booktabs}
\usepackage{siunitx}
\newcommand{\tinytableTabularrayUnderline}[1]{\underline{#1}}
\newcommand{\tinytableTabularrayStrikeout}[1]{\sout{#1}}
\NewTableCommand{\tinytableDefineColor}[3]{\definecolor{#1}{#2}{#3}}

\title{The Cost of Exclusion: Political Discrimination and Trust Deficits in the European Union}

\author[1]{Carlos A. Toruño Paniagua}
\author[2]{...}
\affil[1,2]{\emph{The World Justice Project}}
\date{June 2025}

\begin{document}

\maketitle

\begin{abstract}
Lorem ipsum dolor sit amet, consectetur adipiscing elit. Integer tempus sagittis lobortis. Ut ac ligula a dolor faucibus rutrum. Donec accumsan sem nibh, quis ornare lorem volutpat a. Sed nec efficitur diam. Vivamus molestie turpis lorem, vitae semper lacus convallis sodales. Cras tellus leo, ultrices sit amet fringilla et, aliquam commodo tellus. Maecenas varius tincidunt libero, in aliquam orci iaculis eu. Vestibulum consequat ultrices eros id feugiat. Aliquam scelerisque porta lorem, id venenatis justo. Curabitur feugiat ante nec tellus pulvinar elementum. Orci varius natoque penatibus et magnis dis parturient montes, nascetur ridiculus mus. Sed pellentesque ac dui id accumsan. Nam tellus libero, venenatis nec libero quis, varius rutrum velit. Quisque eu finibus diam. Nunc sit amet massa purus.
\end{abstract}

\section{Introduction}

\section{Literature Review}

\section{Data}

This study uses microdata from general population surveys collected in all 27 European Union Member States as part of the EUROVOICES project, conducted by The World Justice Project. The dataset provides statistically representative coverage across 110 subnational regions throughout the European Union, offering comprehensive geographic representation of the EU population.

The World Justice Project produced subnational information following the framework of territorial divisions established by the Nomenclature of Territorial Units for Statistics (NUTS) system. In each covered country, one level of this system of nested territorial divisions was selected, resulting in a total of 110 subnational regions for which data are presented. It should be noted that the level of statistical representation varies between countries.

The surveys were administered to respondents in 110 regions of the 27 EU Member States. Data collection employed mixed methodologies, with face-to-face interviews conducted in ten countries and online polling utilized in 17 countries. Survey respondents in each country were selected using probability sampling methods designed to ensure representativeness across key sociodemographic criteria, including age, sex, income level, and degree of urbanization (categorized as urban and rural areas). Detailed information on sample sizes for each country, survey methodology employed, data collection periods, and representation levels can be found in Appendix \ref{appendix:a}.

However, due to missing values across the variables of interest, the effective analytical sample for this study is smaller than the total sample published by The World Justice Project. Table \ref{table:1} presents descriptive statistics for the key demographic and political variables among survey respondents included in the final sample.

\begin{table}[hbtp]
\centering
\caption{Sample Demographics and Political Preferences}
\label{table:1}
\footnotesize
\vspace{2mm}
\begin{talltblr}[         %% tabularray outer open
label=none,
note{†} = {Classified as ``\emph{constrained}" if respondents mentioned that their income is not enough fo basic necessities or clothing and ``\emph{unconstrained}" otherwise.},
note{‡} = {Individuals had to answer if they were very interested, interested, a little interested, or not at all interested in politics.},
note{§} = {Political alignment was captured using the respondents' vote intention, where they had to pick a political party to vote for if the general elections were happening that weekend.},
remark{Source} = {Eurovoices General Population Poll 2024},
]                     %% tabularray outer close
{                     %% tabularray inner open
colspec={Q[]Q[]Q[]Q[]Q[]Q[]},
hline{4}={1,2,3,4,5,6}{solid, black, 0.1em},
}                     %% tabularray inner close
\toprule
& Unique & Mean & SD & Min & Max \\ \midrule %% TinyTableHeader
Age & 79 & 49.7 & 16.6 & 18.0 & 99.0 \\
Political Ideology & 11 & 5.4 & 2.3 & 0.0 & 10.0 \\
&  & N & \% &  &  \\
Sex & Female & 22,659 & 51.2 &  &  \\
& Male & 21,576 & 48.8 &  &  \\
Area of Residence & Urban & 31,993 & 72.3 &  &  \\
& Rural & 12,242 & 27.7 &  &  \\
Employment & Employed & 24,821 & 56.1 &  &  \\
& Unemployed/Inactive & 19,414 & 43.9 &  &  \\
Citizenship Status & Citizen & 43,224 & 97.7 &  &  \\
& Foreigner & 1,011 & 2.3 &  &  \\
Marital Status & Married & 26,838 & 60.7 &  &  \\
& Not married & 17,397 & 39.3 &  &  \\
Ethnic Group & Ethnic Majority & 26,629 & 60.2 &  &  \\
& Ethnic Minority & 17,606 & 39.8 &  &  \\
Financial Situation\TblrNote{†} & Constrained & 10,103 & 22.8 &  &  \\
& Unconstrained & 34,132 & 77.2 &  &  \\
Education & Higher Education & 14,278 & 32.3 &  &  \\
& No Higher Education & 29,957 & 67.7 &  &  \\
Interest in Politics\TblrNote{‡} & Interested & 21,887 & 49.5 &  &  \\
& Uninterested & 22,348 & 50.5 &  &  \\
Political Alignment\TblrNote{§} & Incumbent Political Party & 7,742 & 17.5 &  &  \\
& Non-Incumbent Political Party & 36,493 & 82.5 &  &  \\
Political D/H & D/H Experience & 7,002 & 15.8 &  &  \\
& No D/H Experience & 37,233 & 84.2 &  &  \\
\bottomrule
\end{talltblr}
\end{table}


The EUROVOICES survey includes a comprehensive discrimination module that measures 11 different grounds for discrimination: sex, age, disability, ethnicity, migration background, socioeconomic status, religion, and political opinion. Participants indicated whether they experienced discrimination or harassment (D/H) based on any of these characteristics during the 12 months before the survey, and identified where such incidents took place. Since each discrimination ground was assessed independently, individual respondents could report multiple types of discriminatory experiences within the year prior to survey administration.

The survey also includes a comprehensive Trust module asking respondents to rate their trust levels in various public institutions as "a lot," "some," "little," or "no trust." These institutions include local and national government officials, police officers, prosecutors, public defense attorneys, judges, magistrates, political parties, and Parliament members. We recoded these responses into binary indicators, where "a lot" and "some trust" equal 1, and "a little" and "no trust" equal 0. To construct a comprehensive measure of political trust, we created a Trust in Political Institutions Index through Logistic Principal Component Analysis (Landgraf \& Lee, 2020), which aggregates individual trust responses across all measured institutions.

Finally, as displayed in Table \ref{table:1}, the survey not only collects important demographic data but it also asks respondents key political traits such as their interest in politics, vote intention, and it also asks people to rate their political ideology from 0 to 10, where 0 is associated with a far left ideology and 10 is associated with a far right ideology.

\section{Discrimination Experiences in the European Union}

Data from the World Justice Project EUROVOICES household survey reveals that discrimination represents a significant challenge throughout the European Union. In most EU regions, over 25\% of respondents reported experiencing some form of discrimination within the previous year. Table \ref{table:2} displays the prevalence of self-reported discrimination across EU Member States, categorized by the grounds for discrimination.

Political discrimination represents the most frequently reported form of discrimination in 13 of the 27 members, surpassing discrimination based on sex, ethnicity, or migration status. The highest rates are observed in Hungary (34.5\%), Czechia (28.7\%), Slovakia (26.5\%), Austria (25.3\%), and Germany (24.4\%), while Portugal (2.6\%) and Bulgaria (1.5\%) show considerably lower prevalence. Across the EU, an average of 14.7\% of respondents have encountered political discrimination.

One reason why political grounds for discrimination display such high incidence rates across most countries in our sample is that, unlike other grounds such as sex, ethnicity, or religion, political discrimination affects all demographic groups in our sample. While political opinion may not be the predominant form of discrimination experienced by specific minority or marginalized groups, they become the most common ground when considering the entire population. This pattern reflects the universal nature of political discrimination—it can target individuals regardless of their demographic characteristics.

These findings indicate that political discrimination constitutes a prominent and pervasive issue across numerous European contexts. Notably, this form of discrimination has become more prevalent than traditionally recognized categories of discrimination in a substantial number of countries, suggesting an evolution in the nature of exclusionary experiences faced by individuals.

\begin{table}
\centering
\begin{talltblr}[         %% tabularray outer open
entry=none,label=none,
note{}={*Note*: Table displays the percentage of respondents in each country that answered to have had experienced discrimination
  or harrasment for each of the grounds presented to them by the survey.},
]                     %% tabularray outer close
{                     %% tabularray inner open
colspec={Q[]Q[]Q[]Q[]Q[]Q[]Q[]Q[]Q[]},
column{3,4,5,6,7,8,9}={}{halign=c,},
column{1,2}={}{halign=l,},
}                     %% tabularray inner close
\toprule
& Country & Political Opinion & Sex & Gender & Ethnicity & Migration Status & Social Status & Religion \\ \midrule %% TinyTableHeader
& Austria & \num{25.3} & \num{13.0} & \num{9.5} & \num{13.1} & \num{12.6} & \num{17.8} & \num{11.6} \\
& Belgium & \num{13.2} & \num{15.0} & \num{8.5} & \num{13.0} & \num{9.2} & \num{18.1} & \num{10.7} \\
& Bulgaria & \num{1.5} & \num{1.8} & \num{0.7} & \num{2.1} & \num{0.6} & \num{3.1} & \num{1.1} \\
& Croatia & \num{11.8} & \num{8.2} & \num{2.6} & \num{5.0} & \num{3.1} & \num{11.2} & \num{8.3} \\
& Cyprus & \num{11.0} & \num{13.3} & \num{6.1} & \num{6.4} & \num{5.5} & \num{12.3} & \num{6.7} \\
& Czechia & \num{28.7} & \num{13.0} & \num{10.0} & \num{13.9} & \num{16.6} & \num{18.5} & \num{5.4} \\
& Denmark & \num{11.5} & \num{13.0} & \num{7.5} & \num{10.8} & \num{9.9} & \num{14.7} & \num{10.1} \\
& Estonia & \num{15.5} & \num{9.5} & \num{3.4} & \num{6.7} & \num{2.2} & \num{10.1} & \num{2.8} \\
& Finland & \num{13.7} & \num{11.4} & \num{4.6} & \num{4.4} & \num{2.8} & \num{15.9} & \num{5.0} \\
& France & \num{10.4} & \num{11.6} & \num{4.6} & \num{7.7} & \num{4.4} & \num{12.2} & \num{6.7} \\
& Germany & \num{24.4} & \num{11.9} & \num{8.0} & \num{11.1} & \num{10.0} & \num{16.0} & \num{9.5} \\
& Greece & \num{5.2} & \num{3.6} & \num{0.8} & \num{1.5} & \num{1.4} & \num{2.8} & \num{1.0} \\
& Hungary & \num{34.5} & \num{15.9} & \num{16.5} & \num{22.4} & \num{16.5} & \num{27.8} & \num{14.3} \\
& Ireland & \num{12.7} & \num{15.8} & \num{7.1} & \num{9.1} & \num{8.7} & \num{14.8} & \num{9.6} \\
& Italy & \num{13.7} & \num{13.7} & \num{7.8} & \num{8.1} & \num{6.3} & \num{13.9} & \num{8.2} \\
& Latvia & \num{6.0} & \num{2.5} & \num{0.5} & \num{7.5} & \num{1.9} & \num{5.5} & \num{1.9} \\
& Lithuania & \num{5.3} & \num{2.2} & \num{0.1} & \num{2.9} & \num{0.5} & \num{4.9} & \num{1.0} \\
& Malta & \num{13.0} & \num{6.8} & \num{1.6} & \num{4.4} & \num{3.4} & \num{4.4} & \num{4.0} \\
& Netherlands & \num{15.2} & \num{14.0} & \num{9.4} & \num{12.5} & \num{11.2} & \num{15.2} & \num{10.6} \\
& Poland & \num{6.7} & \num{3.0} & \num{1.0} & \num{1.3} & \num{1.0} & \num{3.4} & \num{2.4} \\
& Portugal & \num{2.6} & \num{3.5} & \num{2.6} & \num{4.1} & \num{2.8} & \num{4.1} & \num{3.5} \\
& Romania & \num{4.7} & \num{3.6} & \num{1.8} & \num{3.7} & \num{1.7} & \num{5.3} & \num{2.7} \\
& Slovakia & \num{26.5} & \num{14.1} & \num{10.7} & \num{12.3} & \num{10.1} & \num{18.5} & \num{11.1} \\
& Slovenia & \num{18.1} & \num{10.1} & \num{5.3} & \num{9.2} & \num{7.1} & \num{17.0} & \num{9.0} \\
& Spain & \num{19.8} & \num{14.6} & \num{7.5} & \num{9.4} & \num{8.4} & \num{15.2} & \num{8.3} \\
& Sweden & \num{12.1} & \num{15.3} & \num{5.7} & \num{10.8} & \num{7.2} & \num{13.4} & \num{7.3} \\
\bottomrule
\end{talltblr}
\end{table}


\section{Methodology}

The main objective of this study is to examine whether experiencing discrimination or harassment (D/H) due to political opinions has a causal effect on trust in political institutions. However, a fundamental challenge in causal inference is that assessing the causal effect of these experiences would require observing both potential outcomes in the same individual—those who experienced D/H and those who did not. Since we never observe both potential outcomes for any individual, we must infer the effects of discrimination and harassment by comparing average trust levels between those who experienced such events and those who did not.

The problem is that people who experience political discrimination or harassment might be systematically different from those who do not. For example, individuals who experience D/H events might be more interested in and informed about politics, or display higher levels of civic participation. If these systematic differences correlate with institutional trust levels, then a naïve comparison between the two groups would be misleading. This is known as the selection problem (Angrist \& Pischke, 2009).

Researchers typically turn to randomized controlled experiments—the gold standard for causal inference—to address such questions. However, the nature of this research question makes it both practically difficult and ethically problematic to randomly expose individuals to discrimination or harassment based on their political views. As a result, we adopt quasi-experimental designs to examine the causal effects of political discrimination and harassment on institutional trust [citations needed].

For this study, we implement the matching methodology suggested by Ho et al. (2007) to preprocess our observational data. Our goal is to balance the distribution of covariates between our treatment group (people who experienced D/H events) and control group (people who have not), thereby replicating the randomization achieved in an experimental study (Stuart, 2010).

While regression methods can also address confounding from measured covariates, using regression on matched samples reduces the dependence of our treatment effect estimates on correct model specification. This approach provides more robust causal inferences by making the groups more comparable before applying statistical models.

Importantly, we use matching as a preprocessing method to improve balance between groups, rather than as an imputation method for estimating missing outcomes, as described by Abadie and Imbens (2006, 2016).

We performed nearest neighbor matching\footnote{The matching estimation was performed using the MatchIT R package (Ho et al., 2011).} based on the propensity score, defined as the probability of experiencing a D/H event conditional on a set of observed covariates (Rubin, 1973; Rosenbaum and Rubin, 1983). We estimate this probability using logistic regression. Each unit from the treatment group is matched with the control group unit that has the closest propensity score (1:1 matching). Matching was performed without replacement.\footnote{Matching with replacement means that each control unit can be reused to be matched with any number of treated units if that individual is the closest neighbor to multiple treatment units.} We also establish a region of common support\footnote{The common support region refers to the range of covariate values (or propensity scores) where both treated and control units have sufficient representation.} and discard units that fall outside this region.\footnote{All treatment units fell within the common support region, resulting only in exclusion of individuals from the control group, i.e., people who did not experience political discrimination or harassment.}

The propensity score is estimated using two sets of covariates: demographic variables (geographic region\footnote{Equivalent to one of the 110 NUTS regions in the study.}, sex, age, area of residence, financial situation, education level, employment status, marital status, citizenship status, and ethnic group) and political variables (interest level in politics, political ideology, and political alignment). All covariates and their possible values are listed in Table \ref{table:1}.

Once we have a fully matched study sample, we estimate the effect of having experienced a D/H event on the levels of trust in political institutions through the following specification:

\begin{equation}
Y_{ic} = \alpha + \tau X_{ic} + \mathbf{Z}_{ic}'\mathbf{\beta} + \gamma_c + \varepsilon_{ic}
\end{equation}

Where:
\begin{itemize}
  \item \( Y_{ic} \): political trust for individual \( i \) in region \( c \)
  \item \( X_{ic} \): D/H event (1 = treated, 0 = control)
  \item \( \tau \): treatment effect (ATT)
  \item \( \mathbf{Z}_{ic} \): vector of covariates
  \item \( \mathbf{\beta} \): coefficient vector for covariates
  \item \( \gamma_c \): region fixed effects
  \item \( \varepsilon_{ic} \): error term
\end{itemize}

This specification is estimated using fixed effects OLS regression\footnote{The estimation of the fixed effects was performed using the fixest R package (Bergé, 2018).}. The value of \( \tau \) can be interpreted as the Average Treatment Effect on the Treated (ATT)\footnote{The Average Treatment Effect (ATE) and the Average Treatment Effect on the Treated (ATT) are two foundational metrics in causal inference. ATE represents the average effect of a treatment across the entire population, whereas ATT focuses only on the average effect among individuals who actually received the treatment.} provided that the conditional ignorability (unconfoundedness) assumption holds (Abadie and Cattaneo, 2018; Greifer and Stuart, 2010).

The unconfoundedness assumption requires that treatment assignment be independent of potential outcomes conditional on the set of observed covariates included. However, even when matching achieves balance in observed covariates, unobserved confounders remain a threat to causal identification. Unobserved variables that relate to both experiencing D/H events and trust in political institutions would violate the unconfoundedness assumption and bias our treatment effect estimates. Since conditional ignorability cannot be directly tested, the literature suggests performing sensitivity analyses to assess the plausibility of this assumption and determine how sensitive our estimated effects are to violations of unconfoundedness (Stuart, 2010).

\section{Results}

When we compare trust levels in political institutions between people who have experienced discrimination or harassment (D/H) due to their political opinions and those who have not, we observe that the former group typically achieves lower average scores, as shown in Figure \ref{fig:1}. This figure reveals a strong negative correlation between experiencing D/H events and current levels of institutional trust. However, this relationship cannot be claimed to be causal. As explained previously, people who initially have lower trust in institutions might display higher levels of political participation or activism and, therefore, be more prone to experiencing discrimination.

\begin{figure}[htbp]
\centering
\caption{Trust in Political Institutions by Political D/H Experience}
\label{fig:1}
\includegraphics[width=0.9\textwidth]{"viz/fig_trust_comparison_naive.png"}

\medskip
\justifying\footnotesize 
\textit{Note:} The figure shows the distribution of Trust in Political Institutions Index scores for individuals who experienced political discrimination or harassment (D/H) versus those who did not. Index scores are derived from Logistic PCA of trust responses across nine institutional categories: local authorities, national authorities, police, prosecutors, public defense attorneys, judges, magistrates, political parties, and Parliament members. Red areas show 95\% confidence intervals of median scores. Scores were re-scaled to fit a scale between 1-10.

\textit{Source:} Eurovoices General Population Poll 2024
\end{figure}

We attempt to isolate a causal effect by matching people who have experienced a D/H event with people who have similar demographic characteristics and political traits but have not experienced such events. This approach allows us to approximate the counterfactual outcomes that we cannot directly observe in our data. To evaluate how this matching affects our analysis, we begin by assessing the initial balance in characteristics between people who have experienced D/H events and those who have not.

\subsection{Preprocessing of the study sample using matching methods}

Appendix \ref{appendix:b} shows balance statistics for the whole sample prior to matching. As we can observe, several covariates display standardized mean differences above the thresholds suggested by the literature—0.1 or, more conservatively, 0.05. Of particular note are the high values for average civic participation, interest in politics, and political alignment with the incumbent party, suggesting that the political profiles of people who have suffered D/H due to their political opinions and those who have not are systematically different. Similarly, age and ethnic composition differ significantly between both groups on average, indicating that the demographic composition of the groups is also systematically different. These insights are validated by the variance ratios and empirical cumulative distribution functions (eCDFs).\footnote{The Std. Mean Diff. is the difference in group means standardized to a single scale for all covariates. The literature suggests values below 0.1—or more conservatively, below 0.05—as indicators of good balance (Ali et al., 2011; Stuart et al., 2013). The Var. Ratio compares the variance of a covariate between the treatment and control groups. Ratios close to 1 indicate good balance (Austin, 2009). Empirical Cumulative Distribution Function Statistics (eCDF) assess balance beyond the mean by comparing the full distribution of each covariate across groups (McCaffrey et al., 2004). Values close to zero indicate good balance.}

After performing the matching procedure, we reassess the balance of covariates between the treatment and control groups. The results are presented in Table \ref{table:3}. As we can see, the matching procedure has significantly improved the balance of covariates between both groups. The standardized mean differences for all covariates are now below 0.05, indicating that the treatment and control groups are now comparable in terms of their demographic and political characteristics. This is further supported by the variance ratios, which are now close to 1 for all covariates, and the empirical cumulative distribution function (eCDF) statistics, which are also close to zero for all covariates. Appendix \ref{appendix:c} provides a love plot that visually summarizes the balance of covariates before and after matching.

\begin{table}
\centering
\begin{tabular}[t]{l|r|r|r|r|r|r}
\hline
  & Means Treated & Means Control & Std. Mean Diff. & Var. Ratio & eCDF Mean & eCDF Max\\
\hline
Distance & 0.270 & 0.265 & 0.031 & 1.122 & 0.001 & 0.027\\
\hline
Gender: Female & 0.468 & 0.461 & 0.014 & NA & 0.007 & 0.007\\
\hline
Age & 46.268 & 46.255 & 0.001 & 1.000 & 0.004 & 0.013\\
\hline
Area: Rural & 0.244 & 0.243 & 0.002 & NA & 0.001 & 0.001\\
\hline
Fin. Situation: Constrained & 0.281 & 0.269 & 0.027 & NA & 0.012 & 0.012\\
\hline
Education: Higher Education Diploma & 0.352 & 0.358 & -0.013 & NA & 0.006 & 0.006\\
\hline
Employment: Employed & 0.579 & 0.584 & -0.010 & NA & 0.005 & 0.005\\
\hline
Marital Status: Married & 0.570 & 0.580 & -0.019 & NA & 0.010 & 0.010\\
\hline
Citizenship: Foreigner & 0.025 & 0.025 & 0.002 & NA & 0.000 & 0.000\\
\hline
Ethnic Group: Minority & 0.513 & 0.512 & 0.003 & NA & 0.001 & 0.001\\
\hline
Interest in Politics: High & 0.663 & 0.658 & 0.011 & NA & 0.005 & 0.005\\
\hline
Political Ideology & 5.504 & 5.485 & 0.008 & 1.153 & 0.017 & 0.033\\
\hline
Political Alignment: Incumbent & 0.117 & 0.121 & -0.012 & NA & 0.004 & 0.004\\
\hline
Civic Participation Score & -2.499 & -2.570 & 0.015 & 1.015 & 0.010 & 0.024\\
\hline
\end{tabular}
\end{table}


As a result of the matching procedure, our sample size was reduced from 44,235 individuals to 14,004 individuals. The final matched sample consists of 7,002 individuals who experienced D/H events and 7,002 individuals who did not. This reduction in sample size is a common outcome of matching procedures, as some individuals may not have suitable matches in the control group or may fall outside the common support region. This is the study sample we use to estimate the causal effect of political discrimination and harassment on trust in political institutions.

\subsection{The effect of political discrimination and harassment on trust in political institutions}

When matching achieves balance across observed covariates, it suggests that the treatment and control groups are comparable in ways that validate the assumptions necessary for causal inference (Branson, 2021). However, balance represents a necessary but not sufficient condition for causality, as unobserved confounders may still bias our estimates.

\subsection{Sensitivity analysis}

\subsection{Polarization as the transmission channel}

\section{Discussion}

\newpage
\appendix
\section{Survey methodology and sample characteristics by country}
\label{appendix:a}
\begin{table}
\centering
\begin{tblr}[         %% tabularray outer open
]                     %% tabularray outer close
{                     %% tabularray inner open
colspec={Q[]Q[]Q[]},
column{1}={}{halign=l,},
column{2,3}={}{halign=r,},
}                     %% tabularray inner close
\toprule
Country & N & % \\ \midrule %% TinyTableHeader
Austria & 1525 & 3.4 \\
Belgium & 983 & 2.2 \\
Bulgaria & 997 & 2.3 \\
Croatia & 1561 & 3.5 \\
Cyprus & 810 & 1.8 \\
Czechia & 1348 & 3.0 \\
Denmark & 1519 & 3.4 \\
Estonia & 638 & 1.4 \\
Finland & 1513 & 3.4 \\
France & 3866 & 8.7 \\
Germany & 5639 & 12.7 \\
Greece & 1560 & 3.5 \\
Hungary & 1433 & 3.2 \\
Ireland & 924 & 2.1 \\
Italy & 2750 & 6.2 \\
Latvia & 718 & 1.6 \\
Lithuania & 1105 & 2.5 \\
Malta & 312 & 0.7 \\
Netherlands & 1794 & 4.1 \\
Poland & 4209 & 9.5 \\
Portugal & 983 & 2.2 \\
Romania & 2390 & 5.4 \\
Slovakia & 1519 & 3.4 \\
Slovenia & 672 & 1.5 \\
Spain & 2653 & 6.0 \\
Sweden & 814 & 1.8 \\
\bottomrule
\end{tblr}
\end{table}


\newpage
\section{Pre-Matching Covariate Balance Assessment}
\label{appendix:b}
\begin{table}
\centering
\begin{tabular}[t]{l|r|r|r|r|r|r}
\hline
  & Means Treated & Means Control & Std. Mean Diff. & Var. Ratio & eCDF Mean & eCDF Max\\
\hline
Distance & 0.270 & 0.137 & 0.823 & 1.929 & 0.264 & 0.396\\
\hline
Gender: Female & 0.468 & 0.521 & -0.105 & NA & 0.053 & 0.053\\
\hline
Age & 46.268 & 50.309 & -0.246 & 0.979 & 0.051 & 0.102\\
\hline
Area: Rural & 0.244 & 0.283 & -0.092 & NA & 0.039 & 0.039\\
\hline
Fin. Situation: Constrained & 0.281 & 0.219 & 0.138 & NA & 0.062 & 0.062\\
\hline
Education: Higher Education Diploma & 0.352 & 0.317 & 0.074 & NA & 0.035 & 0.035\\
\hline
Employment: Employed & 0.579 & 0.558 & 0.043 & NA & 0.021 & 0.021\\
\hline
Marital Status: Married & 0.570 & 0.614 & -0.087 & NA & 0.043 & 0.043\\
\hline
Citizenship: Foreigner & 0.025 & 0.022 & 0.015 & NA & 0.002 & 0.002\\
\hline
Ethnic Group: Minority & 0.513 & 0.376 & 0.274 & NA & 0.137 & 0.137\\
\hline
Interest in Politics: High & 0.663 & 0.463 & 0.424 & NA & 0.200 & 0.200\\
\hline
Political Ideology & 5.504 & 5.435 & 0.028 & 1.206 & 0.021 & 0.041\\
\hline
Political Alignment: Incumbent & 0.117 & 0.186 & -0.213 & NA & 0.069 & 0.069\\
\hline
Civic Participation Score & -2.499 & -4.734 & 0.487 & 1.721 & 0.145 & 0.259\\
\hline
\end{tabular}
\end{table}


\newpage
\section{Post-Matching Covariate Balance Assessment: Love Plot}
\label{appendix:c}
\begin{figure}[htbp]
\centering
\includegraphics[width=\textwidth]{"viz/loveplot.png"}

\medskip
\justifying\footnotesize 
\textit{Note:} The figure shows the Absolute Std. Mean Diff. for each covariate before (full sample) and after matching. The dashed and dotted lines represent the thresholds of 0.1 and 0.05, respectively, for good balance. The figure indicates that the matching procedure has significantly improved the balance of covariates between the treatment and control groups.

\textit{Source:} Eurovoices General Population Poll 2024
\end{figure}

\end{document}
